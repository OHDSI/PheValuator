\PassOptionsToPackage{unicode=true}{hyperref} % options for packages loaded elsewhere
\PassOptionsToPackage{hyphens}{url}
%
\documentclass[
]{article}
\usepackage{lmodern}
\usepackage{amssymb,amsmath}
\usepackage{ifxetex,ifluatex}
\ifnum 0\ifxetex 1\fi\ifluatex 1\fi=0 % if pdftex
  \usepackage[T1]{fontenc}
  \usepackage[utf8]{inputenc}
  \usepackage{textcomp} % provides euro and other symbols
\else % if luatex or xelatex
  \usepackage{unicode-math}
  \defaultfontfeatures{Scale=MatchLowercase}
  \defaultfontfeatures[\rmfamily]{Ligatures=TeX,Scale=1}
\fi
% use upquote if available, for straight quotes in verbatim environments
\IfFileExists{upquote.sty}{\usepackage{upquote}}{}
\IfFileExists{microtype.sty}{% use microtype if available
  \usepackage[]{microtype}
  \UseMicrotypeSet[protrusion]{basicmath} % disable protrusion for tt fonts
}{}
\makeatletter
\@ifundefined{KOMAClassName}{% if non-KOMA class
  \IfFileExists{parskip.sty}{%
    \usepackage{parskip}
  }{% else
    \setlength{\parindent}{0pt}
    \setlength{\parskip}{6pt plus 2pt minus 1pt}}
}{% if KOMA class
  \KOMAoptions{parskip=half}}
\makeatother
\usepackage{xcolor}
\IfFileExists{xurl.sty}{\usepackage{xurl}}{} % add URL line breaks if available
\IfFileExists{bookmark.sty}{\usepackage{bookmark}}{\usepackage{hyperref}}
\hypersetup{
  pdftitle={EvaluatingPhenotypeAlgorithms},
  pdfauthor={Joel N. Swerdel},
  pdfborder={0 0 0},
  breaklinks=true}
\urlstyle{same}  % don't use monospace font for urls
\usepackage[margin=1in]{geometry}
\usepackage{color}
\usepackage{fancyvrb}
\newcommand{\VerbBar}{|}
\newcommand{\VERB}{\Verb[commandchars=\\\{\}]}
\DefineVerbatimEnvironment{Highlighting}{Verbatim}{commandchars=\\\{\}}
% Add ',fontsize=\small' for more characters per line
\usepackage{framed}
\definecolor{shadecolor}{RGB}{248,248,248}
\newenvironment{Shaded}{\begin{snugshade}}{\end{snugshade}}
\newcommand{\AlertTok}[1]{\textcolor[rgb]{0.94,0.16,0.16}{#1}}
\newcommand{\AnnotationTok}[1]{\textcolor[rgb]{0.56,0.35,0.01}{\textbf{\textit{#1}}}}
\newcommand{\AttributeTok}[1]{\textcolor[rgb]{0.77,0.63,0.00}{#1}}
\newcommand{\BaseNTok}[1]{\textcolor[rgb]{0.00,0.00,0.81}{#1}}
\newcommand{\BuiltInTok}[1]{#1}
\newcommand{\CharTok}[1]{\textcolor[rgb]{0.31,0.60,0.02}{#1}}
\newcommand{\CommentTok}[1]{\textcolor[rgb]{0.56,0.35,0.01}{\textit{#1}}}
\newcommand{\CommentVarTok}[1]{\textcolor[rgb]{0.56,0.35,0.01}{\textbf{\textit{#1}}}}
\newcommand{\ConstantTok}[1]{\textcolor[rgb]{0.00,0.00,0.00}{#1}}
\newcommand{\ControlFlowTok}[1]{\textcolor[rgb]{0.13,0.29,0.53}{\textbf{#1}}}
\newcommand{\DataTypeTok}[1]{\textcolor[rgb]{0.13,0.29,0.53}{#1}}
\newcommand{\DecValTok}[1]{\textcolor[rgb]{0.00,0.00,0.81}{#1}}
\newcommand{\DocumentationTok}[1]{\textcolor[rgb]{0.56,0.35,0.01}{\textbf{\textit{#1}}}}
\newcommand{\ErrorTok}[1]{\textcolor[rgb]{0.64,0.00,0.00}{\textbf{#1}}}
\newcommand{\ExtensionTok}[1]{#1}
\newcommand{\FloatTok}[1]{\textcolor[rgb]{0.00,0.00,0.81}{#1}}
\newcommand{\FunctionTok}[1]{\textcolor[rgb]{0.00,0.00,0.00}{#1}}
\newcommand{\ImportTok}[1]{#1}
\newcommand{\InformationTok}[1]{\textcolor[rgb]{0.56,0.35,0.01}{\textbf{\textit{#1}}}}
\newcommand{\KeywordTok}[1]{\textcolor[rgb]{0.13,0.29,0.53}{\textbf{#1}}}
\newcommand{\NormalTok}[1]{#1}
\newcommand{\OperatorTok}[1]{\textcolor[rgb]{0.81,0.36,0.00}{\textbf{#1}}}
\newcommand{\OtherTok}[1]{\textcolor[rgb]{0.56,0.35,0.01}{#1}}
\newcommand{\PreprocessorTok}[1]{\textcolor[rgb]{0.56,0.35,0.01}{\textit{#1}}}
\newcommand{\RegionMarkerTok}[1]{#1}
\newcommand{\SpecialCharTok}[1]{\textcolor[rgb]{0.00,0.00,0.00}{#1}}
\newcommand{\SpecialStringTok}[1]{\textcolor[rgb]{0.31,0.60,0.02}{#1}}
\newcommand{\StringTok}[1]{\textcolor[rgb]{0.31,0.60,0.02}{#1}}
\newcommand{\VariableTok}[1]{\textcolor[rgb]{0.00,0.00,0.00}{#1}}
\newcommand{\VerbatimStringTok}[1]{\textcolor[rgb]{0.31,0.60,0.02}{#1}}
\newcommand{\WarningTok}[1]{\textcolor[rgb]{0.56,0.35,0.01}{\textbf{\textit{#1}}}}
\usepackage{graphicx,grffile}
\makeatletter
\def\maxwidth{\ifdim\Gin@nat@width>\linewidth\linewidth\else\Gin@nat@width\fi}
\def\maxheight{\ifdim\Gin@nat@height>\textheight\textheight\else\Gin@nat@height\fi}
\makeatother
% Scale images if necessary, so that they will not overflow the page
% margins by default, and it is still possible to overwrite the defaults
% using explicit options in \includegraphics[width, height, ...]{}
\setkeys{Gin}{width=\maxwidth,height=\maxheight,keepaspectratio}
\setlength{\emergencystretch}{3em}  % prevent overfull lines
\providecommand{\tightlist}{%
  \setlength{\itemsep}{0pt}\setlength{\parskip}{0pt}}
\setcounter{secnumdepth}{5}
% Redefines (sub)paragraphs to behave more like sections
\ifx\paragraph\undefined\else
  \let\oldparagraph\paragraph
  \renewcommand{\paragraph}[1]{\oldparagraph{#1}\mbox{}}
\fi
\ifx\subparagraph\undefined\else
  \let\oldsubparagraph\subparagraph
  \renewcommand{\subparagraph}[1]{\oldsubparagraph{#1}\mbox{}}
\fi

% set default figure placement to htbp
\makeatletter
\def\fps@figure{htbp}
\makeatother


\title{EvaluatingPhenotypeAlgorithms}
\author{Joel N. Swerdel}
\date{2020-03-06}

\begin{document}
\maketitle

{
\setcounter{tocdepth}{3}
\tableofcontents
}
\newpage

\hypertarget{introduction}{%
\section{Introduction}\label{introduction}}

The \texttt{Phevaluator} package enables evaluating the performance
characteristics of phenotype algorithms (PAs) using data from databases
that are translated into the Observational Medical Outcomes Partnership
Common Data Model (OMOP CDM).

This vignette describes how to run the PheValuator process from start to
end in the \texttt{Phevaluator} package.

\hypertarget{overview-of-process}{%
\section{Overview of Process}\label{overview-of-process}}

There are several steps in performing a PA evaluation: 1. Creating the
extremely specific (xSpec), extremely sensitive (xSens), and prevalence
cohorts 2. Creating the Diagnostic Predictive Model using the
PatientLevelPrediction (PLP) package 3. Creating the Evaluation Cohort
4. Creating the Phenotype Algorithms for evaluation 5. Evaluating the
PAs 6. Examining the results of the evaluation

Each of these steps is described in detail below. For this vignette we
will describe the evaluation of PAs for diabetes mellitus (DM).

\hypertarget{creating-the-extremely-specific-xspec-extremely-sensitive-xsens-and-prevalence-cohorts}{%
\subsection{Creating the Extremely Specific (xSpec), Extremely Sensitive
(xSens), and Prevalence
Cohorts}\label{creating-the-extremely-specific-xspec-extremely-sensitive-xsens-and-prevalence-cohorts}}

The extremely specific (xSpec), extremely sensitive (xSens), and
prevalence cohorts are developed using the ATLAS tool. The xSpec is a
cohort where the subjects in the cohort are likely to be positive for
the health outcome of interest (HOI) with a very high probability. This
may be achieved by requiring that subjects have multiple condition codes
for the HOI in their patient record. An
\href{http://www.ohdsi.org/web/atlas/\#/cohortdefinition/1769699}{example}
of this for DM is included in the OHDSI ATLAS repository. In this
example each subject has an initial condition code for DM. The cohort
definition further specifies that each subject also has a second code
for DM between 1 and 30 days after the initial DM code and 10 additional
DM codes in the rest of the patient record. This very specific algorithm
for DM ensures that the subjects in this cohort have a very high
probability for having the condition of DM. This PA also specifies that
subjects are required to have at least 365 days of observation in their
patient record.

An example of an xSens cohort is created by developing a PA that is very
sensitive for the HOI. The system uses the xSens cohort to create a set
of ``noisy'' negative subjects, i.e., subjects with a high likelihood of
not having the HOI. This group of subjects will be used in the model
building process and is described in detail below. An
\href{http://www.ohdsi.org/web/atlas/\#/cohortdefinition/1770120}{example}
of an xSens cohort for DM is also in the OHDSI ATLAS repository.

The system uses the prevalence cohort to provide a reasonable
approximation of the prevalence of the HOI in the population. This
improves the calibration of the predictive model. The system will use
the xSens cohort as the default if a prevalence cohort is not specified.
This group of subjects will be used in the model building process and is
described in detail below. An
\href{http://www.ohdsi.org/web/atlas/\#/cohortdefinition/1770119}{example}
of an prevalence cohort for DM is also in the OHDSI ATLAS repository.

\hypertarget{creating-the-diagnostic-predictive-model}{%
\subsection{Creating the Diagnostic Predictive
Model}\label{creating-the-diagnostic-predictive-model}}

The function createPhenotypeModel develops the diagnostic predictive
model for assessing the probability of having the HOI in the evaluation
cohort.

createPhenotypeModel should have as inputs:

\begin{itemize}
\tightlist
\item
  connectionDetails - connectionDetails created using the function
  createConnectionDetails in the DatabaseConnector package.
\item
  xSpecCohort - The number of the ``extremely specific (xSpec)'' cohort
  definition id in the cohort table (for noisy positives)
\item
  cdmDatabaseSchema - The name of the database schema that contains the
  OMOP CDM instance. Requires read permissions to this database. On SQL
  Server, this should specify both the database and the schema, so for
  example `cdm\_instance.dbo'.
\item
  cohortDatabaseSchema - The name of the database schema that is the
  location where the cohort data used to define the at risk cohort is
  available. If cohortTable = DRUG\_ERA, cohortDatabaseSchema is not
  used by assumed to be cdmSchema. Requires read permissions to this
  database.
\item
  cohortDatabaseTable - The tablename that contains the at risk cohort.
  If cohortTable \textless{}\textgreater{} DRUG\_ERA, then expectation
  is cohortTable has format of COHORT table: cohort\_concept\_id,
  SUBJECT\_ID, COHORT\_START\_DATE, COHORT\_END\_DATE.
\item
  outDatabaseSchema - The name of a database schema where the user has
  write capability. A temporary cohort table will be created here.
\item
  modelOutputFileName - A string designation for the training model file
  Recommended structure: "Model\_(xSpec
  Name)\emph{(CDM)}(Qualifiers)\_(Analysis Date)``,
  e.g.,''Model\_10XStroke\_MyCDM\_Age18-62\_20190101" to designate the
  file was from a \textbf{Model}, built on the \textbf{10 X Stroke}
  xSpec, using the \textbf{MyCDM} database, including ages \textbf{18 to
  62}, and analyzed on \textbf{20190101}.
\item
  xSensCohort - The number of the ``extremely sensitive (xSens)'' cohort
  definition id in the cohort table (used to estimate population
  prevalence and to exclude subjects from the noisy positives)
\item
  prevalenceCohort - The number of the cohort definition id to determine
  the disease prevalence, usually a super-set of the exclCohort
\item
  excludedConcepts - A list of conceptIds to exclude from
  featureExtraction which should include all concept\_ids used to create
  the xSpec and xSens cohorts
\item
  addDescendantsToExclude - Should descendants of excluded concepts also
  be excluded? (default=FALSE)
\item
  mainPopulationCohort - The number of the cohort to be used as a base
  population for the model (default=NULL)
\item
  lowerAgeLimit - The lower age for subjects in the model (default=NULL)
\item
  upperAgeLimit - The upper age for subjects in the model (default=NULL)
\item
  gender - The gender(s) to be included (default c(8507, 8532))
\item
  startDate - The starting date for including subjects in the model
  (default=NULL)
\item
  endDate - The ending date for including subjects in the model
  (default=NULL)
\item
  checkDates - Should observation period dates be checked to guard
  against errors such as dates in the future? (default=TRUE)
\item
  cdmVersion - The CDM version of the database (default=5)
\item
  outFolder - The folder where the output files will be written
  (default=working directory)
\end{itemize}

The createPhenotypeModel function creates a PLP model to be used for
determining the probability of the HOI in the evaluation cohort.

For example:

\begin{Shaded}
\begin{Highlighting}[]
\KeywordTok{options}\NormalTok{(}\DataTypeTok{fftempdir =} \StringTok{"c:/temp/ff"}\NormalTok{) }\CommentTok{#place to store large temporary files}

\NormalTok{connectionDetails <-}\StringTok{ }\KeywordTok{createConnectionDetails}\NormalTok{(}\DataTypeTok{dbms =} \StringTok{"postgresql"}\NormalTok{,}
                                              \DataTypeTok{server =} \StringTok{"localhost/ohdsi"}\NormalTok{,}
                                              \DataTypeTok{user =} \StringTok{"joe"}\NormalTok{,}
                                              \DataTypeTok{password =} \StringTok{"supersecret"}\NormalTok{)}

\NormalTok{phenoTest <-}\StringTok{ }\NormalTok{PheValuator}\OperatorTok{::}\KeywordTok{createPhenotypeModel}\NormalTok{(}\DataTypeTok{connectionDetails =}\NormalTok{ connectionDetails,}
                           \DataTypeTok{xSpecCohort =} \DecValTok{1769699}\NormalTok{,}
                           \DataTypeTok{xSensCohort =} \DecValTok{1770120}\NormalTok{, }
                           \DataTypeTok{cdmDatabaseSchema =} \StringTok{"my_cdm_data"}\NormalTok{,}
                           \DataTypeTok{cohortDatabaseSchema =} \StringTok{"my_results"}\NormalTok{,}
                           \DataTypeTok{cohortDatabaseTable =} \StringTok{"cohort"}\NormalTok{,}
                           \DataTypeTok{outDatabaseSchema =} \StringTok{"scratch.dbo"}\NormalTok{, }\CommentTok{#a database schema with write access}
                           \DataTypeTok{modelOutputFileName =} \StringTok{"Train_10XDM_MyCDM_18-62_20190101"}\NormalTok{,}
                           \DataTypeTok{prevalenceCohort =} \DecValTok{1770119}\NormalTok{, }\CommentTok{#the cohort for prevalence determination}
                           \DataTypeTok{excludedConcepts =} \KeywordTok{c}\NormalTok{(}\DecValTok{201820}\NormalTok{), }
                           \DataTypeTok{addDescendantsToExclude =} \OtherTok{TRUE}\NormalTok{,}
                           \DataTypeTok{mainPopulationCohort =} \DecValTok{0}\NormalTok{, }\CommentTok{#use the entire subject population}
                           \DataTypeTok{lowerAgeLimit =} \DecValTok{18}\NormalTok{, }
                           \DataTypeTok{upperAgeLimit =} \DecValTok{90}\NormalTok{,}
                           \DataTypeTok{startDate =} \StringTok{"20100101"}\NormalTok{,}
                           \DataTypeTok{endDate =} \StringTok{"20171231"}\NormalTok{,}
                           \DataTypeTok{checkDates =} \OtherTok{TRUE}\NormalTok{,}
                           \DataTypeTok{outFolder =} \StringTok{"c:/phenotyping"}\NormalTok{)}
\end{Highlighting}
\end{Shaded}

In this example, we used the cohorts developed in the ``my\_results''
cdm, specifying the location of the cohort table (cohortDatabaseSchema,
cohortDatabaseTable - ``my\_results.cohort'') and where the model will
find the conditions, drug exposures, etc. to inform the model
(cdmDatabaseSchema - ``my\_cdm\_data''). The subjects included in the
model will be those whose first visit in the CDM is between January 1,
2010 and December 31, 2017. We are also specifically excluding the
concept ID 201826, ``Type 2 diabetes mellitus'', which was used to
create the xSpec cohort as well as all of the descendants of that
concept ID. Their ages at the time of first visit will be between 18 and
90. With the parameters above, the name of the predictive model output
from this step will be:
``c:/phenotyping/Train\_10XDM\_MyCDM\_18-62\_20190101.rds''

\hypertarget{creating-the-evaluation-cohort}{%
\subsection{Creating the Evaluation
Cohort}\label{creating-the-evaluation-cohort}}

The function createEvaluationCohort uses the PLP function applyModel to
produce a large cohort of subjects, each with a predicted probability
for the HOI.

createEvaluationCohort should have as inputs:

\begin{itemize}
\tightlist
\item
  connectionDetails - connectionDetails created using the function
  createConnectionDetails in the DatabaseConnector package.
\item
  xSpecCohort - The number of the ``extremely specific (xSpec)'' cohort
  definition id in the cohort table (for noisy positives)
\item
  xSensCohort - The number of the ``extremely sensitive (xSens)'' cohort
  definition id in the cohort table (used to estimate population
  prevalence and to exclude subjects from the noisy positives)
\item
  cdmDatabaseSchema - The name of the database schema that contains the
  OMOP CDM instance. Requires read permissions to this database. On SQL
  Server, this should specifiy both the database and the schema, so for
  example `cdm\_instance.dbo'.
\item
  cohortDatabaseSchema - The name of the database schema that is the
  location where the cohort data used to define the at risk cohort is
  available. Requires read permissions to this database.
\item
  cohortDatabaseTable - The tablename that contains the at risk cohort.
  The expectation is cohortTable has format of COHORT table:
  cohort\_concept\_id, SUBJECT\_ID, COHORT\_START\_DATE,
  COHORT\_END\_DATE.
\item
  outDatabaseSchema - The name of the database schema that is the
  location where the data used to define the outcome cohorts is
  available. Requires read permissions to this database.
\item
  evaluationOutputFileName - A string designation for the evaluation
  cohort file.\\
  Recommended structure: "Evaluation\_(xSpec
  Name)\emph{(CDM)}(Qualifiers)\_(Analysis Date)``,
  e.g.,''Evaluation\_10XStroke\_MyCDM\_Age18-62\_20190101" to designate
  the file was from an \textbf{Evaluation}, built on the \textbf{10 X
  Stroke} xSpec, using the \textbf{MyCDM} database, including ages
  \textbf{18 to 62}, and analyzed on \textbf{20190101}
\item
  modelOutputFileName - A string designation for the training model file
\item
  mainPopulationCohort - The number of the cohort to be used as a base
  population for the model (default=NULL)
\item
  lowerAgeLimit - The lower age for subjects in the model (default=NULL)
\item
  upperAgeLimit - The upper age for subjects in the model (default=NULL)
\item
  startDays - The days to include prior to the cohort start date
  (default=-10000)
\item
  endDays - The days to include after the cohort start date
  (default=10000)
\item
  gender - The gender(s) to be included (default c(8507, 8532))
\item
  startDate - The starting date for including subjects in the model
  (default=NULL)
\item
  endDate - The ending date for including subjects in the model
  (default=NULL)
\item
  cdmVersion - The CDM version of the database (default=5)
\item
  outFolder - The folder where the output files will be written
  (default=working directory)
\item
  savePlpData - Determines whether the large PLP data file is saved.
  Setting this to FALSE will reduce the use of disk space
  (default=FALSE)
\end{itemize}

For example:

\begin{Shaded}
\begin{Highlighting}[]
\KeywordTok{options}\NormalTok{(}\DataTypeTok{fftempdir =} \StringTok{"c:/temp/ff"}\NormalTok{) }\CommentTok{#place to store large temporary files}


\NormalTok{connectionDetails <-}\StringTok{ }\KeywordTok{createConnectionDetails}\NormalTok{(}\DataTypeTok{dbms =} \StringTok{"postgresql"}\NormalTok{,}
                                              \DataTypeTok{server =} \StringTok{"localhost/ohdsi"}\NormalTok{,}
                                              \DataTypeTok{user =} \StringTok{"joe"}\NormalTok{,}
                                              \DataTypeTok{password =} \StringTok{"supersecret"}\NormalTok{)}

\NormalTok{evalCohort <-}\StringTok{ }\NormalTok{PheValuator}\OperatorTok{::}\KeywordTok{createEvaluationCohort}\NormalTok{(}\DataTypeTok{connectionDetails =}\NormalTok{ connectionDetails,}
                              \DataTypeTok{xSpecCohort =} \DecValTok{1769699}\NormalTok{, }
                              \DataTypeTok{xSensCohort =} \DecValTok{1770120}\NormalTok{, }
                              \DataTypeTok{cdmDatabaseSchema =} \StringTok{"my_cdm_data"}\NormalTok{,}
                              \DataTypeTok{cohortDatabaseSchema =} \StringTok{"my_results"}\NormalTok{,}
                              \DataTypeTok{cohortDatabaseTable =} \StringTok{"cohort"}\NormalTok{,}
                              \DataTypeTok{outDatabaseSchema =} \StringTok{"scratch.dbo"}\NormalTok{,}
                              \DataTypeTok{evaluationOutputFileName =} \StringTok{"Eval_10XDM_MyCDM_18-62_20190101"}\NormalTok{,}
                              \DataTypeTok{modelOutputFileName =} \StringTok{"Train_10XDM_MyCDM_18-62_20190101"}\NormalTok{,}
                              \DataTypeTok{mainPopulationCohort =} \DecValTok{0}\NormalTok{, }
                              \DataTypeTok{lowerAgeLimit =} \DecValTok{18}\NormalTok{, }
                              \DataTypeTok{upperAgeLimit =} \DecValTok{90}\NormalTok{,}
                              \DataTypeTok{startDate =} \StringTok{"20100101"}\NormalTok{,}
                              \DataTypeTok{endDate =} \StringTok{"20171231"}
                              \DataTypeTok{outFolder =} \StringTok{"c:/phenotyping"}\NormalTok{,}
                              \DataTypeTok{savePlpData =} \OtherTok{FALSE}\NormalTok{)}
\end{Highlighting}
\end{Shaded}

In this example, the parameters specify that the function should use the
model file: ``c:/phenotyping/Train\_10XDM\_MyCDM\_18-62\_20190101.rds''
to produce the evaluation cohort file:
``c:/phenotyping/Eval\_10XDM\_MyCDM\_18-62\_20190101.rds'' The
evaluation cohort file above will be used the evaluation of the PAs
provided in the next step.

\hypertarget{creating-the-phenotype-algorithms-for-evaluation}{%
\subsection{Creating the Phenotype Algorithms for
evaluation}\label{creating-the-phenotype-algorithms-for-evaluation}}

The next step is to create the PAs to be evaluated. These are specific
to the research question of interest. For certain questions, a very
sensitive algorithm may be required; others may require a very specific
algorithm. For this example, we will test an algorithm which requires
that the subject have a diagnosis code for DM from an in-patient setting
where the code was specified as the primary reason for discharge. An
\href{http://www.ohdsi.org/web/atlas/\#/cohortdefinition/1769702}{example}
of this algorithm is in the OHDSI ATLAS repository. The output of this
function is a list containing 2 data frames, one with the results of the
PA evaluation and a second with a set of subject IDs that were
determined to be true positives, false positives, or false negatives
based on prediction threshold of 50\%. A true positive, with this
criteria, would be a subject that was included in the PA and also had a
predicted value for the HOI of 0.5 or greater. A false positive would be
a subject who was included in the PA and whose predicted probability was
less than 0.5. A false negative would be a subject who was not included
in the PA but had a predicted probability of the HOI or 0.5 or greater.

testPhenotypeAlgorithm should have as inputs:

\begin{itemize}
\tightlist
\item
  connectionDetails - ConnectionDetails created using the function
  createConnectionDetails in the DatabaseConnector package.
\item
  cutPoints - A list of threshold predictions for the evaluations.
  Include ``EV'' for the expected value
\item
  evaluationOutputFileName - The full file name with path for the
  evaluation file
\item
  phenotypeCohortId - The number of the cohort of the phenotype
  algorithm to test
\item
  phenotypeText - A string to identify the phenotype algorithm in the
  output file
\item
  cdmShortName - A string to identify the CDM tested (Default = NULL)
\item
  order - The order of this algorithm for sorting in the output file
  (used when there are multiple phenotypes to test) (Default = 1)
\item
  modelText - Descriptive name for the model (Default = NULL)
\item
  xSpecCohort - The number of the ``extremely specific (xSpec)'' cohort
  definition id in the cohort table (for noisy positives) (Default =
  NULL)
\item
  xSensCohort - The number of the ``extremely sensitive (xSens)'' cohort
  definition id in the cohort table (used to exclude subjects from the
  base population) (Default = NULL)
\item
  prevalenceCohort - The number of the cohort definition id to determine
  the disease prevalence, (default=xSensCohort)
\item
  cohortDatabaseSchema - The name of the database schema that is the
  location where the cohort data used to define the at risk cohort is
  available. Requires read permissions to this database.
\item
  cohortTable - The tablename that contains the at risk cohort. The
  expectation is cohortTable has format of COHORT table:
  cohort\_concept\_id, SUBJECT\_ID, COHORT\_START\_DATE,
  COHORT\_END\_DATE.
\item
  washoutPeriod - The washoutPeriod is used when testing algorithms
  where there is an enforced prior observation period before the index
  date
\end{itemize}

For example:

\begin{Shaded}
\begin{Highlighting}[]
\KeywordTok{options}\NormalTok{(}\DataTypeTok{fftempdir =} \StringTok{"c:/temp/ff"}\NormalTok{) }\CommentTok{#place to store large temporary files}

\NormalTok{connectionDetails <-}\StringTok{ }\KeywordTok{createConnectionDetails}\NormalTok{(}\DataTypeTok{dbms =} \StringTok{"postgresql"}\NormalTok{,}
                                              \DataTypeTok{server =} \StringTok{"localhost/ohdsi"}\NormalTok{,}
                                              \DataTypeTok{user =} \StringTok{"joe"}\NormalTok{,}
                                              \DataTypeTok{password =} \StringTok{"supersecret"}\NormalTok{)}

\NormalTok{phenoResult <-}\StringTok{ }\NormalTok{PheValuator}\OperatorTok{::}\KeywordTok{testPhenotypeAlgorithm}\NormalTok{(}\DataTypeTok{connectionDetails =}\NormalTok{ connectionDetails,}
               \DataTypeTok{cutPoints =} \KeywordTok{c}\NormalTok{(}\StringTok{"EV"}\NormalTok{),}
               \DataTypeTok{evaluationOutputFileName =} \StringTok{"c:/phenotyping/lr_results_Eval_10X_DM_MyCDM.rds"}\NormalTok{,}
               \DataTypeTok{phenotypeCohortId =} \DecValTok{1769702}\NormalTok{,}
               \DataTypeTok{cdmShortName =} \StringTok{"myCDM"}\NormalTok{,}
               \DataTypeTok{phenotypeText =} \StringTok{"All Diabetes by Phenotype 1 X In-patient, 1st Position"}\NormalTok{,}
               \DataTypeTok{order =} \DecValTok{1}\NormalTok{,}
               \DataTypeTok{modelText =} \StringTok{"Diabetes Mellitus xSpec Model - 10 X T2DM"}\NormalTok{,}
               \DataTypeTok{xSpecCohort =} \DecValTok{1769699}\NormalTok{,}
               \DataTypeTok{xSensCohort =} \DecValTok{1770120}\NormalTok{,}
               \DataTypeTok{prevalenceCohort =} \DecValTok{1770119}\NormalTok{,}
               \DataTypeTok{cohortDatabaseSchema =} \StringTok{"my_results"}\NormalTok{,}
               \DataTypeTok{cohortTable =} \StringTok{"cohort"}\NormalTok{,}
               \DataTypeTok{washoutPeriod =} \DecValTok{0}\NormalTok{)}
\end{Highlighting}
\end{Shaded}

In this example, we are using only the expected value (``EV''). Given
that parameter setting, the output from this step will provide
performance characteristics (i.e, sensitivity, specificity, etc.) at
each prediction threshold as well as those using the expected value
calculations as described in the \href{vignettes/Figure2.png}{Step 2
diagram}. The evaluation uses the prediction information for the
evaluation cohort developed in the prior step. The data frames produced
from this step may be saved to a csv file for detailed examination.

\end{document}
