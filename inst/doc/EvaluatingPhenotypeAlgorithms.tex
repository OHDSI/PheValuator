\PassOptionsToPackage{unicode=true}{hyperref} % options for packages loaded elsewhere
\PassOptionsToPackage{hyphens}{url}
%
\documentclass[
]{article}
\usepackage{lmodern}
\usepackage{amssymb,amsmath}
\usepackage{ifxetex,ifluatex}
\ifnum 0\ifxetex 1\fi\ifluatex 1\fi=0 % if pdftex
  \usepackage[T1]{fontenc}
  \usepackage[utf8]{inputenc}
  \usepackage{textcomp} % provides euro and other symbols
\else % if luatex or xelatex
  \usepackage{unicode-math}
  \defaultfontfeatures{Scale=MatchLowercase}
  \defaultfontfeatures[\rmfamily]{Ligatures=TeX,Scale=1}
\fi
% use upquote if available, for straight quotes in verbatim environments
\IfFileExists{upquote.sty}{\usepackage{upquote}}{}
\IfFileExists{microtype.sty}{% use microtype if available
  \usepackage[]{microtype}
  \UseMicrotypeSet[protrusion]{basicmath} % disable protrusion for tt fonts
}{}
\makeatletter
\@ifundefined{KOMAClassName}{% if non-KOMA class
  \IfFileExists{parskip.sty}{%
    \usepackage{parskip}
  }{% else
    \setlength{\parindent}{0pt}
    \setlength{\parskip}{6pt plus 2pt minus 1pt}}
}{% if KOMA class
  \KOMAoptions{parskip=half}}
\makeatother
\usepackage{xcolor}
\IfFileExists{xurl.sty}{\usepackage{xurl}}{} % add URL line breaks if available
\IfFileExists{bookmark.sty}{\usepackage{bookmark}}{\usepackage{hyperref}}
\hypersetup{
  pdftitle={EvaluatingPhenotypeAlgorithms},
  pdfauthor={Joel N. Swerdel},
  pdfborder={0 0 0},
  breaklinks=true}
\urlstyle{same}  % don't use monospace font for urls
\usepackage[margin=1in]{geometry}
\usepackage{color}
\usepackage{fancyvrb}
\newcommand{\VerbBar}{|}
\newcommand{\VERB}{\Verb[commandchars=\\\{\}]}
\DefineVerbatimEnvironment{Highlighting}{Verbatim}{commandchars=\\\{\}}
% Add ',fontsize=\small' for more characters per line
\usepackage{framed}
\definecolor{shadecolor}{RGB}{248,248,248}
\newenvironment{Shaded}{\begin{snugshade}}{\end{snugshade}}
\newcommand{\AlertTok}[1]{\textcolor[rgb]{0.94,0.16,0.16}{#1}}
\newcommand{\AnnotationTok}[1]{\textcolor[rgb]{0.56,0.35,0.01}{\textbf{\textit{#1}}}}
\newcommand{\AttributeTok}[1]{\textcolor[rgb]{0.77,0.63,0.00}{#1}}
\newcommand{\BaseNTok}[1]{\textcolor[rgb]{0.00,0.00,0.81}{#1}}
\newcommand{\BuiltInTok}[1]{#1}
\newcommand{\CharTok}[1]{\textcolor[rgb]{0.31,0.60,0.02}{#1}}
\newcommand{\CommentTok}[1]{\textcolor[rgb]{0.56,0.35,0.01}{\textit{#1}}}
\newcommand{\CommentVarTok}[1]{\textcolor[rgb]{0.56,0.35,0.01}{\textbf{\textit{#1}}}}
\newcommand{\ConstantTok}[1]{\textcolor[rgb]{0.00,0.00,0.00}{#1}}
\newcommand{\ControlFlowTok}[1]{\textcolor[rgb]{0.13,0.29,0.53}{\textbf{#1}}}
\newcommand{\DataTypeTok}[1]{\textcolor[rgb]{0.13,0.29,0.53}{#1}}
\newcommand{\DecValTok}[1]{\textcolor[rgb]{0.00,0.00,0.81}{#1}}
\newcommand{\DocumentationTok}[1]{\textcolor[rgb]{0.56,0.35,0.01}{\textbf{\textit{#1}}}}
\newcommand{\ErrorTok}[1]{\textcolor[rgb]{0.64,0.00,0.00}{\textbf{#1}}}
\newcommand{\ExtensionTok}[1]{#1}
\newcommand{\FloatTok}[1]{\textcolor[rgb]{0.00,0.00,0.81}{#1}}
\newcommand{\FunctionTok}[1]{\textcolor[rgb]{0.00,0.00,0.00}{#1}}
\newcommand{\ImportTok}[1]{#1}
\newcommand{\InformationTok}[1]{\textcolor[rgb]{0.56,0.35,0.01}{\textbf{\textit{#1}}}}
\newcommand{\KeywordTok}[1]{\textcolor[rgb]{0.13,0.29,0.53}{\textbf{#1}}}
\newcommand{\NormalTok}[1]{#1}
\newcommand{\OperatorTok}[1]{\textcolor[rgb]{0.81,0.36,0.00}{\textbf{#1}}}
\newcommand{\OtherTok}[1]{\textcolor[rgb]{0.56,0.35,0.01}{#1}}
\newcommand{\PreprocessorTok}[1]{\textcolor[rgb]{0.56,0.35,0.01}{\textit{#1}}}
\newcommand{\RegionMarkerTok}[1]{#1}
\newcommand{\SpecialCharTok}[1]{\textcolor[rgb]{0.00,0.00,0.00}{#1}}
\newcommand{\SpecialStringTok}[1]{\textcolor[rgb]{0.31,0.60,0.02}{#1}}
\newcommand{\StringTok}[1]{\textcolor[rgb]{0.31,0.60,0.02}{#1}}
\newcommand{\VariableTok}[1]{\textcolor[rgb]{0.00,0.00,0.00}{#1}}
\newcommand{\VerbatimStringTok}[1]{\textcolor[rgb]{0.31,0.60,0.02}{#1}}
\newcommand{\WarningTok}[1]{\textcolor[rgb]{0.56,0.35,0.01}{\textbf{\textit{#1}}}}
\usepackage{graphicx,grffile}
\makeatletter
\def\maxwidth{\ifdim\Gin@nat@width>\linewidth\linewidth\else\Gin@nat@width\fi}
\def\maxheight{\ifdim\Gin@nat@height>\textheight\textheight\else\Gin@nat@height\fi}
\makeatother
% Scale images if necessary, so that they will not overflow the page
% margins by default, and it is still possible to overwrite the defaults
% using explicit options in \includegraphics[width, height, ...]{}
\setkeys{Gin}{width=\maxwidth,height=\maxheight,keepaspectratio}
\setlength{\emergencystretch}{3em}  % prevent overfull lines
\providecommand{\tightlist}{%
  \setlength{\itemsep}{0pt}\setlength{\parskip}{0pt}}
\setcounter{secnumdepth}{5}
% Redefines (sub)paragraphs to behave more like sections
\ifx\paragraph\undefined\else
  \let\oldparagraph\paragraph
  \renewcommand{\paragraph}[1]{\oldparagraph{#1}\mbox{}}
\fi
\ifx\subparagraph\undefined\else
  \let\oldsubparagraph\subparagraph
  \renewcommand{\subparagraph}[1]{\oldsubparagraph{#1}\mbox{}}
\fi

% set default figure placement to htbp
\makeatletter
\def\fps@figure{htbp}
\makeatother


\title{EvaluatingPhenotypeAlgorithms}
\author{Joel N. Swerdel}
\date{2020-04-21}

\begin{document}
\maketitle

{
\setcounter{tocdepth}{3}
\tableofcontents
}
\newpage

\hypertarget{introduction}{%
\section{Introduction}\label{introduction}}

The \texttt{Phevaluator} package enables evaluating the performance
characteristics of phenotype algorithms (PAs) using data from databases
that are translated into the Observational Medical Outcomes Partnership
Common Data Model (OMOP CDM).

This vignette describes how to run the PheValuator process from start to
end in the \texttt{Phevaluator} package.

\hypertarget{overview-of-process}{%
\section{Overview of Process}\label{overview-of-process}}

There are several steps in performing a PA evaluation: 1. Creating the
extremely specific (xSpec), extremely sensitive (xSens), and prevalence
cohorts 2. Creating the Diagnostic Predictive Model and the Evaluation
Cohort using the PatientLevelPrediction (PLP) package 5. Evaluating the
PAs 6. Examining the results of the evaluation

Each of these steps is described in detail below. For this vignette we
will describe the evaluation of PAs for diabetes mellitus (DM).

\hypertarget{creating-the-extremely-specific-xspec-extremely-sensitive-xsens-and-prevalence-cohorts}{%
\subsection{Creating the Extremely Specific (xSpec), Extremely Sensitive
(xSens), and Prevalence
Cohorts}\label{creating-the-extremely-specific-xspec-extremely-sensitive-xsens-and-prevalence-cohorts}}

The extremely specific (xSpec), extremely sensitive (xSens), and
prevalence cohorts are developed using the ATLAS tool. The xSpec is a
cohort where the subjects in the cohort are likely to be positive for
the health outcome of interest (HOI) with a very high probability. This
may be achieved by requiring that subjects have multiple condition codes
for the HOI in their patient record. An
\href{http://www.ohdsi.org/web/atlas/\#/cohortdefinition/1769699}{example}
of this for DM is included in the OHDSI ATLAS repository. In this
example each subject has an initial condition code for DM. The cohort
definition further specifies that each subject also has a second code
for DM between 1 and 30 days after the initial DM code and 10 additional
DM codes in the rest of the patient record. This very specific algorithm
for DM ensures that the subjects in this cohort have a very high
probability for having the condition of DM. This PA also specifies that
subjects are required to have at least 365 days of observation in their
patient record.

An example of an xSens cohort is created by developing a PA that is very
sensitive for the HOI. The system uses the xSens cohort to create a set
of ``noisy'' negative subjects, i.e., subjects with a high likelihood of
not having the HOI. This group of subjects will be used in the model
building process and is described in detail below. An
\href{http://www.ohdsi.org/web/atlas/\#/cohortdefinition/1770120}{example}
of an xSens cohort for DM is also in the OHDSI ATLAS repository.

The system uses the prevalence cohort to provide a reasonable
approximation of the prevalence of the HOI in the population. This
improves the calibration of the predictive model. The system will use
the xSens cohort as the default if a prevalence cohort is not specified.
This group of subjects will be used in the model building process and is
described in detail below. An
\href{http://www.ohdsi.org/web/atlas/\#/cohortdefinition/1770119}{example}
of an prevalence cohort for DM is also in the OHDSI ATLAS repository.

\hypertarget{evaluating-phenotype-algorithms-for-health-conditions}{%
\subsection{Evaluating phenotype algorithms for health
conditions}\label{evaluating-phenotype-algorithms-for-health-conditions}}

The function createEvaluationCohort creates a diagnostic predictive
model and an evaluation cohort that will allow the user to perform an
analysis for determining the performance characteristics for one or more
phenotype algorithms (cohort definitions) for health conditions. This
function initiates the process for the first two steps in PheValuator,
namely:\\
1) Develop a diagnostic predictive model for the health condition.\\
2) Select a large, random set of subjects from the dataset and use the
model to determine the probability of each of the subjects having the
health condition.

createEvaluationCohort should have as inputs:

\begin{itemize}
\tightlist
\item
  connectionDetails - connectionDetails created using the function
  createConnectionDetails in the DatabaseConnector package
\item
  oracleTempSchema - A schema where temp tables can be created in Oracle
  (default==NULL)
\item
  xSpecCohortId - The number of the ``extremely specific (xSpec)''
  cohort definition id in the cohort table (for noisy positives)
\item
  xSensCohortId - The number of the ``extremely sensitive (xSens)''
  cohort definition id in the cohort table (used to estimate population
  prevalence and to exclude subjects from the noisy positives)
\item
  prevalenceCohortId - The number of the cohort definition id to
  determine the disease prevalence, usually a super-set of the
  exclCohort
\item
  cdmDatabaseSchema - The name of the database schema that contains the
  OMOP CDM instance. Requires read permissions to this database. On SQL
  Server, this should specify both the database and the schema, so for
  example `cdm\_instance.dbo'.
\item
  cohortDatabaseSchema - The name of the database schema that is the
  location where the cohort data used to define the at risk cohort is
  available. If cohortTable = DRUG\_ERA, cohortDatabaseSchema is not
  used by assumed to be cdmSchema. Requires read permissions to this
  database.
\item
  cohortTable - The tablename that contains the at risk cohort. If
  cohortTable \textless{}\textgreater{} DRUG\_ERA, then expectation is
  cohortTable has format of COHORT table: cohort\_concept\_id,
  SUBJECT\_ID, COHORT\_START\_DATE, COHORT\_END\_DATE.
\item
  workDatabaseSchema - The name of a database schema where the user has
  write capability. A temporary cohort table will be created here.
\item
  covariateSettings - There are two choices for this setting depending
  on the type of halth outcome to be analyzed:

  \begin{enumerate}
  \def\labelenumi{\arabic{enumi})}
  \tightlist
  \item
    For chronic health outcomes, supply the function
    \textbf{createDefaultChronicCovariateSettings()} with the parameters
    for this function being:\\
    a) excludedCovariateConceptIds - A list of conceptIds to exclude
    from featureExtraction which should include all concept\_ids used to
    create the xSpec and xSens cohorts\\
    b) addDescendantsToExclude - Should descendants of excluded concepts
    also be excluded? (default=FALSE)\\
  \item
    For acute health outcomes, supply the function
    \textbf{createDefaultAcuteCovariateSettings()} with the parameters
    for this function the same as in the function for chronic health
    conditions, namely:\\
    a) excludedCovariateConceptIds - A list of conceptIds to exclude
    from featureExtraction which should include all concept\_ids used to
    create the xSpec and xSens cohorts\\
    b) addDescendantsToExclude - Should descendants of excluded concepts
    also be excluded? (default=FALSE)\\
  \end{enumerate}
\item
  mainPopulationCohortId - The number of the cohort to be used as a base
  population for the model (default=0)
\item
  mainPopulationCohortIdStartDay - When specifying a
  mainPopulationCohortId, the number of days relative to the
  mainPopulationCohortId cohort start date to begin including visits.
  (default=0, i.e., 0 days before the start of the
  mainPopulationCohortId cohort start date)
\item
  mainPopulationCohortIdEndtDay - When specifying a
  mainPopulationCohortId, the number of days relative to the
  mainPopulationCohortId cohort start date to end including visits.
  (default=0, i.e., 0 days after the start of the mainPopulationCohortId
  cohort start date)
\item
  baseSampleSize - The maximum number of subjects in the evaluation
  cohort (default=2M)
\item
  lowerAgeLimit - The lower age for subjects in the model (default=NULL)
\item
  upperAgeLimit - The upper age for subjects in the model (default=NULL)
\item
  visitLength - The minimum length of index visit for noisy negative
  comparison for acute health conditions (default=3 days). As a
  guideline, use the average length of a hospital visit for those with
  the health condition.
\item
  gender - The gender(s) to be included (default c(8507, 8532))
\item
  startDate - The starting date for including subjects in the model
  (default=NULL)
\item
  endDate - The ending date for including subjects in the model
  (default=NULL)
\item
  cdmVersion - The CDM version of the database (default=5)
\item
  outFolder - The folder where the output files will be written
  (default=working directory)
\item
  evaluationCohortId - A string designation for the evaluation cohort
  file (default = ``main'')\\
\item
  removeSubjectsWithFutureDates - Should observation end dates be
  checked to remove future dates (default=TRUE)
\item
  savePlpData - Determines whether the large PLP data file is saved.
  Setting this to FALSE will reduce the use of disk space
  (default=FALSE)
\item
  modelType - The type of health condition to be analyzed either
  ``chronic'' (default) or ``acute''
\end{itemize}

The createEvaluationCohort function will produce the following
artifacts:\\
1) A Patient Level Prediction file (in .rds format) containg the
information from the model building process\\
2) A Patient Level Prediction file (in .rds format) containg the
information from applying the model to the evaluation cohort

For example:

\begin{Shaded}
\begin{Highlighting}[]
\KeywordTok{options}\NormalTok{(}\DataTypeTok{fftempdir =} \StringTok{"c:/temp/ff"}\NormalTok{) }\CommentTok{#place to store large temporary files}

\NormalTok{connectionDetails <-}\StringTok{ }\KeywordTok{createConnectionDetails}\NormalTok{(}\DataTypeTok{dbms =} \StringTok{"postgresql"}\NormalTok{,}
                                              \DataTypeTok{server =} \StringTok{"localhost/ohdsi"}\NormalTok{,}
                                              \DataTypeTok{user =} \StringTok{"joe"}\NormalTok{,}
                                              \DataTypeTok{password =} \StringTok{"supersecret"}\NormalTok{)}

\NormalTok{phenoTest <-}\StringTok{ }\KeywordTok{createEvaluationCohort}\NormalTok{(}\DataTypeTok{connectionDetails =}\NormalTok{ connectionDetails,}
                                   \DataTypeTok{xSpecCohortId =} \DecValTok{1769699}\NormalTok{,}
                                   \DataTypeTok{xSensCohortId =} \DecValTok{1770120}\NormalTok{,}
                                   \DataTypeTok{prevalenceCohortId =} \DecValTok{1770119}\NormalTok{,}
                                   \DataTypeTok{cdmDatabaseSchema =} \StringTok{"my_cdm_data"}\NormalTok{,}
                                   \DataTypeTok{cohortDatabaseSchema =} \StringTok{"my_results"}\NormalTok{,}
                                   \DataTypeTok{cohortTable  =} \StringTok{"cohort"}\NormalTok{,}
                                   \DataTypeTok{workDatabaseSchema =} \StringTok{"scratch.dbo"}\NormalTok{,}
                                   \DataTypeTok{covariateSettings =} 
                                    \KeywordTok{createDefaultChronicCovariateSettings}\NormalTok{(}
                                     \DataTypeTok{excludedCovariateConceptIds =} \KeywordTok{c}\NormalTok{(}\DecValTok{201826}\NormalTok{),}
                                     \DataTypeTok{addDescendantsToExclude =} \OtherTok{TRUE}\NormalTok{),}
                                   \DataTypeTok{baseSampleSize =} \DecValTok{2000000}\NormalTok{,}
                                   \DataTypeTok{lowerAgeLimit =} \DecValTok{18}\NormalTok{,}
                                   \DataTypeTok{upperAgeLimit =} \DecValTok{90}\NormalTok{,}
                                   \DataTypeTok{gender =} \KeywordTok{c}\NormalTok{(}\DecValTok{8507}\NormalTok{, }\DecValTok{8532}\NormalTok{),}
                                   \DataTypeTok{startDate =} \StringTok{"20101010"}\NormalTok{,}
                                   \DataTypeTok{endDate =} \StringTok{"21000101"}\NormalTok{,}
                                   \DataTypeTok{cdmVersion =} \StringTok{"5"}\NormalTok{,}
                                   \DataTypeTok{outFolder =} \StringTok{"c:/phenotyping"}\NormalTok{,}
                                   \DataTypeTok{evaluationCohortId =} \StringTok{"diabetes"}\NormalTok{,}
                                   \DataTypeTok{removeSubjectsWithFutureDates =} \OtherTok{TRUE}\NormalTok{,}
                                   \DataTypeTok{saveEvaluationCohortPlpData =} \OtherTok{FALSE}\NormalTok{,}
                                   \DataTypeTok{modelType =} \StringTok{"chronic"}\NormalTok{)}
\end{Highlighting}
\end{Shaded}

In this example, we used the cohorts developed in the ``my\_results''
cdm, specifying the location of the cohort table (cohortDatabaseSchema,
cohortTable - ``my\_results.cohort'') and where the model will find the
conditions, drug exposures, etc. to inform the model (cdmDatabaseSchema
- ``my\_cdm\_data''). The subjects included in the model will be those
whose first visit in the CDM is between January 1, 2010 and December 31,
2017. We are also specifically excluding the concept ID 201826, ``Type 2
diabetes mellitus'', which was used to create the xSpec cohort as well
as all of the descendants of that concept ID. Their ages at the time of
first visit will be between 18 and 90.

In this example, the parameters specify that the function will create
the model file:\\
``c:/phenotyping/model\_diabetes.rds'',

produce the evaluation cohort file:\\
``c:/phenotyping/evaluationCohort\_diabetes.rds''

\hypertarget{evaluating-the-phenotype-algorithms-to-be-used-in-studies}{%
\subsection{Evaluating the phenotype algorithms to be used in
studies}\label{evaluating-the-phenotype-algorithms-to-be-used-in-studies}}

The function testPhenotypeAlgorithm allows the user to determine the
performance characteristics of phenotype algorithms (cohort defintions)
to be used in studies. It uses the evaluation cohort developed in the
previous step. The same evaluation cohort may be used to test as many
different phenotype algorithms as you wish that pertain to the same
health condition.

testPhenotypeAlgorithm should the following parameters:

\begin{itemize}
\tightlist
\item
  connectionDetails - connectionDetails created using the function
  createConnectionDetails in the DatabaseConnector package
\item
  cutPoints - A list of threshold predictions for the evaluations.
  Include ``EV'' for the expected value
\item
  outFolder - The folder where the cohort evaluation output files are
  written
\item
  evaluationCohortId - A string used to generate the file names for the
  evaluation cohort. This will be the same name used in
  createEvaluationCohort described above.
\item
  cdmDatabaseSchema - The name of the database schema that contains the
  OMOP CDM instance. Requires read permissions to this database. On SQL
  Server, this should specifiy both the database and the schema, so for
  example `cdm\_instance.dbo'.
\item
  cohortDatabaseSchema - The name of the database schema that is the
  location where the cohort data used to define the at risk cohort is
  available. Requires read permissions to this database.
\item
  cohortTable - The tablename that contains the at risk cohort. The
  expectation is cohortTable has format of COHORT table:
  cohort\_concept\_id, SUBJECT\_ID, COHORT\_START\_DATE,
  COHORT\_END\_DATE.
\item
  phenotypeCohortId - The ID of the cohort to evaluate in the specified
  cohort table.
\item
  washoutPeriod - The mininum required continuous observation time prior
  to index date for subjects within the cohort to test (Default = 0).
  For example:
\end{itemize}

\begin{Shaded}
\begin{Highlighting}[]
\KeywordTok{options}\NormalTok{(}\DataTypeTok{fftempdir =} \StringTok{"c:/temp/ff"}\NormalTok{) }\CommentTok{#place to store large temporary files}

\NormalTok{connectionDetails <-}\StringTok{ }\KeywordTok{createConnectionDetails}\NormalTok{(}\DataTypeTok{dbms =} \StringTok{"postgresql"}\NormalTok{,}
                                              \DataTypeTok{server =} \StringTok{"localhost/ohdsi"}\NormalTok{,}
                                              \DataTypeTok{user =} \StringTok{"joe"}\NormalTok{,}
                                              \DataTypeTok{password =} \StringTok{"supersecret"}\NormalTok{)}

\NormalTok{phenotypeResults <-}\StringTok{ }\KeywordTok{testPhenotypeAlgorithm}\NormalTok{(connectionDetails,}
                                   \DataTypeTok{cutPoints =} \KeywordTok{c}\NormalTok{(}\StringTok{"EV"}\NormalTok{),}
                                   \DataTypeTok{outFolder =} \StringTok{"c:/phenotyping"}\NormalTok{,}
                                   \DataTypeTok{evaluationCohortId =} \StringTok{"diabetes"}\NormalTok{,}
                                   \DataTypeTok{phenotypeCohortId =} \DecValTok{7142}\NormalTok{,}
                                   \DataTypeTok{cdmDatabaseSchema =} \StringTok{"my_cdm_data"}\NormalTok{,}
                                   \DataTypeTok{cohortDatabaseSchema =} \StringTok{"my_results"}\NormalTok{,}
                                   \DataTypeTok{cohortTable  =} \StringTok{"cohort"}\NormalTok{,}
                                   \DataTypeTok{washoutPeriod =} \DecValTok{365}\NormalTok{)}
\end{Highlighting}
\end{Shaded}

In this example, we are using only the expected value (``EV''). Given
that parameter setting, the output from this step will provide
performance characteristics (i.e, sensitivity, specificity, etc.) at
each prediction threshold as well as those using the expected value
calculations as described in the \href{vignettes/Figure3.png}{Step 2
diagram}. The evaluation uses the prediction information for the
evaluation cohort developed in the prior step. This function returns a
dataframe with the performance characteristics of the phenotype
algorithm that was tested. The user can write this dataframe to a csv
file using code such as:

\begin{Shaded}
\begin{Highlighting}[]
      \KeywordTok{write.csv}\NormalTok{(phenotypeResults, }\StringTok{"c:/phenotyping/diabetes_results.csv"}\NormalTok{, }\DataTypeTok{row.names =} \OtherTok{FALSE}\NormalTok{)}
\end{Highlighting}
\end{Shaded}

\end{document}
